\documentclass[lang = cn, color = black, 10pt]{AllThatStax}
\title{万智牌锁牌名录}
\subtitle{AllThatStax}

\author{FeonixY@StaxMan}
\date{July 25, 2024——BLB}
\version{Alpha}

\extrainfo{全世界的锁子哥应当联合起来。—— 锁子哥}

\setcounter{tocdepth}{3}

\logo{X-.jpg}
\cover{cover.png}

% 本文档命令
\usepackage{array}

% 修改合法性颜色
\definecolor{legal}{RGB}{117, 152, 110}
\definecolor{notlegal}{RGB}{174, 174, 174}
\definecolor{banned}{RGB}{204, 125, 131}
\definecolor{restricted}{RGB}{128, 167, 182}

% 修改标题页的色带
\definecolor{customcolor}{RGB}{20,84,156}
\colorlet{coverlinecolor}{customcolor}

\begin{document}

\maketitle
\frontmatter

\chapter*{序章}
\markboth{序章}{序章}

本文档收录万智牌(Magic: The Gathering)中所有的锁牌。

在本文档的定义下,锁牌指具有锁效应的永久物牌。锁效应指可以阻碍一位或多位牌手(这里的牌手必须包括至少一位对手,否则不被认定为是锁效应,下面不再赘述)进行某种操作的触发式异能、静止式异能或替代式效应。

需要注意的是,所阻碍的操作必须是“一般意义上正常游戏进程会执行的操作”,具体来说,包括以下操作:

\begin{itemize}
	\item 保持正常的游戏步骤
	\item 保持正常的异能或效应
	\item 重置永久物
	\item 使由自己操控的永久物保持在战场上
	\item 抓牌
	\item 执行特殊动作
	\item 使永久物保持对自己最有利的状态
	\item 搜寻牌库
	\item 起动起动式异能
	\item 施放咒语
	\item 触发触发式异能
	\item 攻击/格挡
	\item 和坟墓场互动
	\item 维持手牌数量
\end{itemize}

也就是说,帷幕,辟邪,守护及其他类似异能,不被认为是锁效应。

关于锁效应,还有三点需要强调:

\begin{enumerate}
	\item 其一,锁效应需要是一个“全局”的效果,而不能只影响某个特定的物件(注意是某个而不是某类),例如穿髓金针就具有锁效应,但提莎娜的缚潮师就不具有锁效应
	\item 其二,锁效应不需要其控制者为了使锁效应生效付出额外代价。注意付出代价的条件是使锁效应生效而不是维持锁牌本身保持于战场上,所以类似累计维持的效果便不属于这种情况,这同时也不意味着锁效应不能需要其控制者付出额外代价。
	\item 其三,锁效应要么阻止受影响的牌手执行某种操作,要么使其为了执行这个操作而付出一定代价。这个代价一定是会直接导致受影响的牌手输掉游戏的效果,具体来说,这些代价包括:
		\begin{itemize}
			\item 失去生命
			\item 支付法术力
			\item 横置、跃离、牺牲、消灭或放逐由自己操控的永久物
			\item 磨牌
			\item 弃牌
			\item 将由自己操控的永久物移回手上或牌库中
		\end{itemize}
\end{enumerate}

以下对锁牌的各种分类方式进行详细介绍。

\section*{按照软硬分类}

按照锁牌对阻碍进行操作的程度,可以将锁牌分成硬锁和软锁两类。

\begin{itemize}
    \item 硬锁:锁效应可以阻止一位或多位牌手执行某种操作的锁牌。
    \item 软锁:锁效应可以使一位或多位牌手在执行某种操作时付出某种代价的锁牌,有些锁效应的代价是必须支付的,这类锁效应称为I类软锁;有些锁效应的代价是选择性支付的,通常会在选择不支付时对受影响的牌手施加某种负面效果,这类锁效应称为II类软锁。
\end{itemize}

\section*{按照所锁操作分类}

按照针对锁效应所阻碍的操作,可将锁牌分为以下几类:

\begin{itemize}
	\item 冬之球效应:使某一类永久物不能重置或者不能重置多于一定数量的牌,例如该锁类别的名称“冬之球” \hfill \textbf{(重置永久物)}
    \item 生物Hate:该类牌最常见的效应为“具有某种特征的生物-X/-X”;还包括一些具有修改生物的攻防或异能效应的牌,例如“谦卑” \hfill \textbf{(使由自己操控的永久物保持在战场上,使永久物保持对自己最有利的状态)}
	\item 地Hate:多数牌具有改变地所产的法术力的类别的效应,例如“腥红之月”,还包括一些会摧毁地的效应 \hfill \textbf{(使由自己操控的永久物保持在战场上,使永久物保持对自己最有利的状态)}
	\item 抓牌锁:限制抓牌的牌,例如“启示天灾希欧蕊” \hfill \textbf{(抓牌)}
	\item 眠梦法球效应:使某一类永久物需横置进入战场的效应,例如该锁类别的名称“眠梦法球” \hfill \textbf{(使永久物保持对自己最有利的状态)}
	\item 搜寻牌库锁:限制搜寻牌库的牌,例如“狮族仲裁者” \hfill \textbf{(搜寻牌库)}
	\item 起动式异能锁:限制起动式异能的牌,例如“万创卡恩” \hfill \textbf{(起动起动式异能)}
	\item 释放锁:限制释放某一类咒语的牌,例如“卓尼斯官员” \hfill \textbf{(施放咒语)}
	\item ETB(Enter The Battlefield)锁:限制永久物进战场触发异能的牌,例如“迟钝法球” \hfill \textbf{(触发触发式异能)}
	\item 攻击/格挡锁:限制进攻/格挡的牌,例如“魂魅拘禁” \hfill \textbf{(攻击/格挡)}
	\item 坟墓场Hate:限制与坟场进行互动的牌,包括阻止某些牌进入坟场,例如“道西虚空行者”和“虚空地脉”,也包括阻止坟墓场中的牌成为目标或者离开坟墓场等,例如“挖坟人囚笼”和“虔敬新兵丹尼克” \hfill \textbf{(和坟墓场互动)}
	\item 手牌Hate:限制手牌数量的牌,包括减少手牌数量上限和弃牌,例如“核心卜算师金吉塔厦”和“夜幕本殿” \hfill \textbf{(维持手牌数量)}
	\item 获得生命锁:阻止获得生命的牌,例如“莽闯暴猛龙” \hfill \textbf{(保持正常的异能或效应)}
	\item 依法治理效应:限制每回合施放咒语的数量的牌,例如该锁类别的名称“依法治理” \hfill \textbf{(施放咒语)}
	\item 大礼拜堂效应:使某一类永久物获得”即在维持开始时,不支付一定法术力就牺牲该永久物“异能的牌,例如该锁效应类别的名称“潘卓欧谷大礼拜堂” \hfill \textbf{(使由自己操控的永久物保持在战场上)}
	\item 烟囱效应:在维持开始时牺牲某一类永久物的效应,例如该锁类别的名称“烟囱” \hfill \textbf{(使由自己操控的永久物保持在战场上)}
	\item 其他:限制其他上述未提到操作的牌,例如“卡洛夫看守犬” \hfill \textbf{(保持正常的游戏步骤,执行特殊动作)}
\end{itemize}

本书中锁牌按照法术力费用和牌张类别进行分类,每张牌都会标出其中文基本信息、锁牌类型和赛制合法性。

对于类似“神器生物”这种多类别,按照“生物>神器>结界>其他”的顺序进行分类。

\tableofcontents

\mainmatter

\chapter{1费}

\section{生物}

\section{神器}

\section{结界}

\section{其他}

\chapter{2费}

\section{生物}

\section{神器}

\section{结界}

\section{其他}

\chapter{3费}

\section{生物}

\section{神器}

\section{结界}

\section{其他}

\chapter{4费}

\section{生物}

\section{神器}

\section{结界}

\section{其他}

\chapter{5费}

\section{生物}

\section{神器}

\section{结界}

\section{其他}

\chapter{6费}

\section{生物}

\section{神器}

\section{结界}

\section{其他}

\chapter{7+费}

\section{生物}

\section{神器}

\section{结界}

\section{其他}

\chapter{0费(包括地)}

\end{document}
